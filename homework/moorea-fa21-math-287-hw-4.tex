\documentclass[12pt,oneside]{amsart}

\title{Math 287 Homework 4}
\author{Andrew Moore}
\date{September 24, 2021} % the due date of the homework

\usepackage[T1]{fontenc}
\usepackage{amsmath,amsfonts,amssymb,amsthm}
\usepackage[letterpaper,margin=1.5in]{geometry}
\usepackage{fancyhdr}
\pagestyle{fancy}
\usepackage{enumitem}

% Extra space between lines
\linespread{2.4}

\theoremstyle{remark}
\newtheorem{exer}{Exercise}
\newtheorem{claim}{Claim}[exer]

\newcommand{\bfN}{\mathbf{N}}
\newcommand{\bfZ}{\mathbf{Z}}

\newenvironment{answer}{\bigskip\noindent\emph{Answer.}}{\hfill$\diamond$\newline}

\begin{document}
\maketitle

%
%
% Question 1
%
%

\newpage
\begin{exer}
Explain the proof of Proposition 4.6(ii). The textbook gives a proof of Proposition 4.6(ii). Rewrite the proof in more detail and with more explanation.
\end{exer}

\begin{proof}
I assume that we're already satisfied with the proof of Proposition 4.6(i). For Proposition 4.6(ii), we're asked to prove $b^mb^k = b^{m + k}$. Let $P(k)$ denote the statement $b^mb^k = b^{m + k}$. We will prove $P(k)$ using induction, assuming $k \geq 0$. We'll start by assuming that $b \in \bfZ$, and that $m \in \bfZ_{\geq 0}$.

We will first examine the base case $P(k = 0)$. By definition, we know that $b^0 = 1$. This means we can rewrite $P(k = 0)$ as:
\begin{align*}
b^m \cdot 1 &= b^m \cdot b^0 &=
\end{align*}
\end{proof}

%
%
% Question 2
%
%

\newpage
\begin{exer}
Proposition 4.7(iii). For all $k \in \bfN$, $10^k + 3 \cdot 4^{k+2} + 5$ is divisible by $9$.
\end{exer}

\begin{proof}
Let $P(k)$ represent the statement $10^k + 3 \cdot 4^{k+2} + 5$ is divisible by $9$. We are asked to prove that the statement is true for all $k \in \bfN$.  We will prove this statement by induction. Beginning with 1 as a base case, we have $P(k = 1)$:
\begin{align*}
10^k + 3 \cdot 4^{k+2} + 5 &= 10^1 + 3 \cdot 4^{1 + 2} + 5 \\
                           &= 10 + 3 \cdot 4^3 + 5 \\
                           &= 10 + 3 \cdot 64 + 5 \\
                           &= 10 + 192 + 5 \\
                           &= 207 \\
                           &= 23 \cdot 9.
\end{align*}
We have shown that the statement is true for at least one $k$ in $\bfN$. Let us assume that $P(k)$ is true, i.e., $10^k + 3 \cdot 4^{k+2} + 5 = 9y$ where $y \in \bfZ$. We will now attempt to show that $P(k + 1)$ is true:
\begin{align*}
10^{k + 1} + 3 \cdot 4^{k + 1 + 2} + 5 &= 10^{k + 1} + 3 \cdot 4^{k + 2 + 1} + 5 \\
                                       &= 10^k \cdot 10^1 + 3 \cdot 4^{k + 2} \cdot 4^1 + 5 \\
                                       &= 10^1 \cdot 10^k + 3 \cdot 4^{k + 2} \cdot 4^1 + 5 \\
                                       &= 10^1 \cdot 10^k + 3 \cdot 4^k \cdot 4^3 + 5 \\
                                       &= 10^1 \cdot 10^k + 192 \cdot 4^k + 5
\end{align*}

\end{proof}

%
%
% Question 3
%
%

\newpage
\begin{exer}
Project 4.9.

In this problem you will (1) determine for which natural numbers the statement is true, and (2) prove your answer. In your answer, you should state very clearly which natural numbers make the statement true: something like “For all natural numbers $k$ such that (your answer here), $k^2 < 2^k$.” Or you could phrase it differently, for example, as “If $k$ is (your answer here), $k^2 < 2^k$.” You will have to fill in what condition is needed for the $k$. Then, prove your statement.

To find the right condition, please try some $k$ values. Try  $k = 1, 2, 3, ...$ (we are talking about natural numbers so it makes sense to count up from 1). Which $k$ values make
$k^2 < 2^k$ true?
\end{exer}

\begin{answer}
Trying a few values of $k$, such as $k = 1, 2, 3, 4, 5, 6, 7$, we have

\begin{align*}
1^2 &< 2^1 \Rightarrow 1 < 2 \tag{k = 1, true} \\
2^2 &< 2^2 \Rightarrow 4 < 4 \tag{k = 2, false} \\
3^2 &< 2^3 \Rightarrow 9 < 8 \tag{k = 3, false} \\
4^2 &< 2^4 \Rightarrow 16 < 16 \tag{k = 4, false} \\
5^2 &< 2^5 \Rightarrow 25 < 32 \tag{k = 5, true} \\
6^2 &< 2^6 \Rightarrow 36 < 64 \tag{k = 6, true} \\
7^2 &< 2^7 \Rightarrow 49 < 128 \tag{k = 7, true}
\end{align*}

Based on these initial observations, I would claim that $k^2 < 2^k$ is true if $k \in \bfN$ is greater than or equal to 5. We'll set aside the trivial exception of $k = 1$.

\begin{proof}
Let $P(k)$ denote the statement $k^2 < 2^k$. We will use induction to prove that $P(k)$ is true for all $k \in \bfN$, such that $k \geq 5$. First, as a base case, we will show that $P(k = 5)$ is true:
\begin{align*}
5^2 &< 2^5 \Rightarrow 25 < 32.
\end{align*}
Assuming $P(k)$ is true for some $k \geq 5$, we will now show that $P(k + 1)$ is true. Starting on the left-hand side, we have:
\begin{align*}
(k + 1)^2 &= k^2 + 2k + 1 \\
          &< k^2 + k^2 \tag{$\because 2k + 1 < k^2$, $\forall k \in \bfN : k \geq 3$; see sub-proof below} \\
          &= 2(k^2) \tag{re-expressing addition as multiplication} \\
          &< 2(2^k) \tag{the inductive hypothesis, $k^2 < 2^k$} \\
          &= 2^{k + 1}. \tag{via Proposition 4.6(ii)}
\end{align*}

\emph{Sub-proof.} Above we claimed that $2k + 1 < k^2$ for all $k \in \bfN$ such that $k \geq 3$. We will refer to this claim as $Q(k)$, and will prove it using induction. Starting with the base case $Q(3)$, we have: $2(3) + 1 < (3)^2 \Rightarrow 7 < 9$. Assuming $Q(k)$ is true for some $k \geq 3$, we will show $Q(k + 1)$ is also true:
\begin{align*}
2(k + 1) + 1 &< (k + 1)^2 \\
      2k + 3 &< k^2 + 2k + 1 \tag{distributing on both sides} \\
           2 &< k^2. \tag{subtracting out common terms}
\end{align*}
We know that k is at least 3, and that $2 < 3^2$. As $k$ increases infinitely, this statement will remain true. Therefore $2k + 1 < k^2$ for $k \geq 3$. $\blacksquare$

Having shown this underlying claim as true, we are able to conclude that $(k + 1)^2 < 2^{k + 1}$, thus finishing our induction. % $(k + 1)^2 = k^2 + (2k + 1) < k^2 + k^2 < 2^k + 2^k = 2(2^k) = 2^{k + 1}$.
\end{proof}

\end{answer}

%
%
% Question 4
%
%

\newpage
\begin{exer}
Find $\sum_{j = 0}^k f_j$, where the $f_j$ are Fibonacci numbers as defined in the textbook. Prove your answer.

Your answer will have a clear statement: $\sum_{j = 0}^k f_j =$ (your answer). Then, a proof of your answer.

Hint: Try $\sum_{j = 0}^k f_j$ for several values of $k$ (e.g., $k = 1, 2, 3, ..., 6, ...$). Look for a pattern. This is “experimental mathematics”, where you try some things, gather data, and look for a pattern!

For your proof, use induction.
\end{exer}

\end{document}
