\documentclass[12pt,oneside]{amsart}

\title{Math 287 Homework 3}
\author{Andrew Moore}
\date{September 17, 2021} % the due date of the homework

\usepackage[T1]{fontenc}
\usepackage{amsmath,amsfonts,amssymb,amsthm}
\usepackage[letterpaper,margin=1.5in]{geometry}
\usepackage{fancyhdr}
\pagestyle{fancy}
\usepackage{enumitem}

% Extra space between lines
\linespread{2.4}

\theoremstyle{remark}
\newtheorem{exer}{Exercise}
\newtheorem{claim}{Claim}[exer]

\newcommand{\bfN}{\mathbf{N}}
\newcommand{\bfZ}{\mathbf{Z}}

\newenvironment{answer}{\bigskip\noindent\emph{Answer.}}{\hfill$\diamond$\newline}

\begin{document}
\maketitle

%
%
% Question 1
%
%

\newpage
\begin{exer}
Project 3.1. Express each of the following statements using quantifiers.
\begin{enumerate}[label={(\roman*)},start={3}]
\item Every integer is the product of two integers.
\item The equation $x^2-2y^2 = 3$ has an integer solution.
\end{enumerate}
\end{exer}

\begin{answer}
\begin{enumerate}[label={(\roman*)},start={3}]
\item $\forall m \in \bfZ, \exists a \in \bfZ$ and $\exists b \in \bfZ$, such that $m = a \cdot b$.
\item $\exists x \in \bfZ$ and $\exists y \in \bfZ$, such that $x^2 - 2y^2 = 3$.
\end{enumerate}
\end{answer}

%
%
% Question 2
%
%

\newpage
\begin{exer}
Project 3.2. In each of the following cases explain what is meant by the statement and decide whether it is true or false.
\begin{enumerate}[label={(\roman*)},start={3}]
\item For each $x \in \mathbf{Z}$ there exists $y \in \mathbf{Z}$ such that $xy=x$.
\item There exists $y \in \mathbf{Z}$ such that for each $x \in \mathbf{Z}$, $xy=x$.
\end{enumerate}
\end{exer}

\begin{answer}
\begin{enumerate}[label={(\roman*)},start={3}]
\item This case states that for every integer $m$ in the set of $\bfZ$, another intger $y$ exists that satisfies the equation $xy = x$. I would say that it is true, but it is not specific enough. We know there exists only one $y$ that satisfies the equation $xy = x$, which is $y = 1$ (Axiom 1.3).
\item This case states that there is an integer $y$, and for each integer $m$ in $\bfZ$, $y$ satisfies the equation $xy = x$. As above, I would say this statement is true, but should be made stronger by saying "there exists a unique integer $y = 1$, which satisfies the equation $xy = x$". Alternatively, the (strong) statement could be expressed as $\exists!y \in \bfZ \colon xy = x$.
\end{enumerate}
\end{answer}

%
%
% Question 3
%
%

\newpage
\begin{exer}
Project 3.7. Negate the following statements.
\begin{enumerate}[label={(\roman*)},start={4}]
\item The newspaper article was neither accurate nor entertaining. (Phrased differently: "The newspaper article was not accurate, and not entertaining.")
\item If $\gcd(m,n)$ is odd, then $m$ or $n$ is odd.
\item H/N is a normal subgroup of G/N if and only if H is a normal subgroup of G.
\item For each $\varepsilon > 0$ there exists $N \in \mathbf{N}$ such that
for all $n \geq N$, $|a_n-L| < \varepsilon$.
\end{enumerate}
\end{exer}

\begin{answer}
\begin{enumerate}[label={(\roman*)},start={4}]
\item The newspaper article was either accurate, or entertaining.
\item If $gcd(m, n)$ is even, then $m$ and $n$ are even.
\item H/N is a normal subgroup of G/N and H is not a normal subgroup of G, \emph{or} H is a normal subgroup of G and H/N is not a normal subgroup of G/N.
\item There exists some $\varepsilon > 0$, for all $N$ in $\bfN$, such that an $n \geq N$ exists where $|a_n - L|$ is greater than or equal to $\varepsilon$. \\

% $(\exists \varepsilon > 0, \forall N \in \bfN)$ such that $(\exists n \geq N, |a_n - L| \geq \varepsilon$).
\end{enumerate}
\end{answer}

%
%
% Question 4
%
%

\newpage
\begin{exer}
For all $k \in \mathbf{N}$, $5^k + 3$ is divisible by $4$.
\begin{enumerate}
\item Write what this statement says for $k=1$. Is it true or false? Explain.
\item Write what this statement says for $k=2$. Is it true or false? Explain.
\item Write what this statement says for $k=3$. Is it true or false? Explain.
\end{enumerate}
Now, prove the statement for all $k \in \mathbf{N}$, using induction.
\end{exer}

\begin{enumerate}
\item For $k=1$, the statement says that $5^1+3$ is divisible by $4$.
It's true because $5+3=8$, and $8 = 4 \cdot 2$.
\item For $k = 2$, the statement says that $5^2 + 3$ is divisible by $4$. It's true because $25 + 3 = 28$, and $28 = 4 \cdot 7$.
\item For $k = 3$ the statement says that $5^3 + 3$ is divisible by $4$. It's true because $125 + 3 = 128$, and $128 = 4 \cdot 32$.
\end{enumerate}
\begin{proof}
We will use induction on $k$. Let $P(k)$ denote the statement

$5^k + 3$ is divisible by $4$.

The induction principle states we must check $P(1)$, i.e., the base case ($k = 1$):
\begin{align*}
5^1 + 3 &= 5 + 3 \\
          &= 8 \\
          &= 4 \cdot 2.
\end{align*}

Next we will assume that $P(k)$ is true for some $k \in \bfN$, and show that $P(n + 1)$ also holds. To assume that $5^k + 3$ is divisible by four is to state that there exists some $y \in \bfZ$ such that

$5^k + 3 = 4y$.

We now need to show that $5^{k + 1} + 3 = 4z$ for some $z \in \bfZ$. That is

$5^{k + 1} + 3 = (5^k \cdot 5^1) + 3 = 5 \cdot 5^k + 3 = 4z$.

We can rewrite the left-hand side as:

$5 \cdot (4y) = 4(5y) = 4z$.

By associativity, we can set $z = 5y$, which we know to be an integer (because $y$ and $z$ are integers). Thus we have shown that there exists a $z \in \bfZ$, such that $5^{k + 1} + 3 = 4z$. This concludes our induction.
\end{proof}

\end{document}
