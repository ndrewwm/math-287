\documentclass[12pt,oneside]{amsart}

\title{Math 287 Homework 3}
\author{Andrew Moore}
\date{September 17, 2021} % the due date of the homework

\usepackage[T1]{fontenc}
\usepackage{amsmath,amsfonts,amssymb,amsthm}
\usepackage[letterpaper,margin=1.5in]{geometry}
\usepackage{fancyhdr}
\pagestyle{fancy}
\usepackage{enumitem}

% Extra space between lines
\linespread{2.4}

\theoremstyle{remark}
\newtheorem{exer}{Exercise}
\newtheorem{claim}{Claim}[exer]

\newcommand{\bfN}{\mathbf{N}}
\newcommand{\bfZ}{\mathbf{Z}}

\newenvironment{answer}{\bigskip\noindent\emph{Answer.}}{\hfill$\diamond$\newline}

\begin{document}
\maketitle

\newpage
\begin{exer}
Project 3.1. Express each of the following statements using quantifiers.
\begin{enumerate}[label={(\roman*)},start={3}]
\item Every integer is the product of two integers.
\item The equation $x^2-2y^2 = 3$ has an integer solution.
\end{enumerate}
\end{exer}

\begin{answer}
\begin{enumerate}[label={(\roman*)},start={3}]
\item -enter your answer here-
\item -enter your answer here-
\end{enumerate}
\end{answer}

\newpage
\begin{exer}
Project 3.2. In each of the following cases explain what is meant by the statement
and decide whether it is true or false.
\begin{enumerate}[label={(\roman*)},start={3}]
\item For each $x \in \mathbf{Z}$ there exists $y \in \mathbf{Z}$ such that $xy=x$.
\item There exists $y \in \mathbf{Z}$ such that for each $x \in \mathbf{Z}$, $xy=x$.
\end{enumerate}
\end{exer}

\begin{answer}
\begin{enumerate}[label={(\roman*)},start={3}]
\item -enter your answer here-
\item -enter your answer here-
\end{enumerate}
\end{answer}

\newpage
\begin{exer}
Project 3.7. Negate the following statements.
\begin{enumerate}[label={(\roman*)},start={4}]
\item The newspaper article was neither accurate nor entertaining.
\item If $\gcd(m,n)$ is odd, then $m$ or $n$ is odd.
\item (you can enter this one)
\item For each $\varepsilon > 0$ there exists $N \in \mathbf{N}$ such that
for all $n \geq N$, $|a_n-L| < \varepsilon$.
\end{enumerate}
\end{exer}

\begin{answer}
\begin{enumerate}[label={(\roman*)},start={4}]
\item -enter your answer here-
\item -enter your answer here-
\item
\item
\end{enumerate}
\end{answer}

\newpage
\begin{exer}
For all $k \in \mathbf{N}$, $5^k + 3$ is divisible by $4$.
\begin{enumerate}
\item Write what this statement says for $k=1$. Is it true or false? Explain.
\item Write what this statement says for $k=2$. Is it true or false? Explain.
\item Write what this statement says for $k=3$. Is it true or false? Explain.
\end{enumerate}
Now, prove the statement for all $k \in \mathbf{N}$, using induction.
\end{exer}

\begin{proof}
\begin{enumerate}
\item For $k=1$, the statement says that $5^1+3$ is divisible by $4$.
It's true because $5+3=8$, and $8 = 4 \cdot 2$.
\item
\end{enumerate}
-enter your proof here-
\end{proof}

\end{document}
