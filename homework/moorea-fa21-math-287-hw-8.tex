\documentclass[12pt,oneside]{amsart}

\title{Math 287 Homework 8}
\author{Andrew Moore}
\date{November 6, 2021} % the due date of the homework

\usepackage[T1]{fontenc}
\usepackage{amsmath,amsfonts,amssymb,amsthm}
\usepackage[letterpaper,margin=1.5in]{geometry}
\usepackage{fancyhdr}
\pagestyle{fancy}
\usepackage{enumitem}

% Extra space between lines
\linespread{1.4}

\theoremstyle{remark}
\newtheorem{exer}{Exercise}
\newtheorem{claim}{Claim}[exer]

\newcommand{\bfN}{\mathbf{N}}
\newcommand{\bfZ}{\mathbf{Z}}

\newenvironment{answer}{\bigskip\noindent\emph{Answer.}}{\hfill$\diamond$\newline}

\begin{document}
\maketitle

\begin{exer}
We need to find the determinants of these matrices:

\[
  \begin{bmatrix}
    1 & 1 \\
    1 & 2
  \end{bmatrix},
  \begin{bmatrix}
    1 & 2 \\
    2 & 3
  \end{bmatrix},
  \begin{bmatrix}
    2 & 3 \\
    3 & 5
  \end{bmatrix},
  \begin{bmatrix}
    3 & 5 \\
    5 & 8
  \end{bmatrix},
  \begin{bmatrix}
    5 & 8 \\
    8 & 13
  \end{bmatrix}
\]
\end{exer}

\begin{answer}
The determinants are

\[
  \det \begin{bmatrix}
    1 & 1 \\
    1 & 2
  \end{bmatrix} = 1,
\]

\[
  \det \begin{bmatrix}
    1 & 2 \\
    2 & 3
  \end{bmatrix} = -1,
\]

\[
  \det \begin{bmatrix}
    2 & 3 \\
    3 & 5
  \end{bmatrix} = 1,
\]

\[
  \det \begin{bmatrix}
    3 & 5 \\
    5 & 8
  \end{bmatrix} = -1,
\]

\[
  \det \begin{bmatrix}
    5 & 8 \\
    8 & 13
  \end{bmatrix} = 1.
\]
\end{answer}

\newpage
\begin{exer}
Proposition 9.7(ii): If $f: A \to B$ is surjective and $g: B \to C$ is surjective, then $g \circ f: A \to C$ is surjective.
% LaTeX: The composition sign is made with \circ, for example $g \circ f$ makes .
%
% Note: A proof for this was given in class. For the homework assignment you can find a proof yourself, or you can rewrite and explain the class proof in your own words.
\end{exer}

\begin{proof}
Assume that $f: A \to B$ and $g: B \to C$ are surjective functions. We intend to show that $g \circ f: A \to C$ is also surjective. That is \[ \exists c \in C, \forall a \in A: (g \circ f)(a) = c. \] $g \circ f$ is a composition of $f$ and $g$, defined as \[ g(f(a)) \text{ for all } a \in A. \]

By hypothesis, we know that $f$ maps the entirety of $A$'s elements ($f$ is surjective). This means \[ \exists b \in B, \forall a \in A \text{ such that } f(a) = b. \] We also know that $g$ maps the entirety of B's elements ($g$ is surjective). That is \[ \exists c \in C, \forall b \in B \text{ such that } g(b) = c. \] Because $f \circ g$ is a composition, inputs ($a$'s) are evaluated first by $f$, and the subsequent outputs ($b$'s) are then fed as new inputs for $g$. This means, we can take any arbitrary element $b \in B$, and know that there is an $a \in A$ that satifies the following equation: $b = f(a)$. Thus we can write

\begin{equation}
\begin{split}
         g(b) &= c \\
      g(f(a)) &= c \\
(g \circ f)(a) &= c.
\end{split}
\end{equation}
This means for every $a \in A$, there must be an $c \in C$ that corresponds to it. Therefore, $g \circ f$ must be surjective. This concludes the proof.
\end{proof}

\newpage
\begin{exer}
Prove the claims:

For any $n \geq 2$, if $f_1, f_2, \dotsc, f_n$ are each injective, then $f_1 \circ f_2 \circ \dotsb \circ f_n$ is injective.

For any $n \geq 2$, if $f_1, f_2, \dotsc, f_n$ are each surjective, then $f_1 \circ f_2 \circ \dotsb \circ f_n$ is surjective.

For any $n \geq 2$, if $f_1, f_2, \dotsc, f_n$ are each bijective, then $f_1 \circ f_2 \circ \dotsb \circ f_n$ is bijective.
%
% (Hint: Use induction and Proposition 9.7.)
\end{exer}

\begin{claim}
For any $n \geq 2$, if $f_1, f_2, \dotsc, f_n$ are each injective, then $f_1 \circ f_2 \circ \dotsb \circ f_n$ is injective.
\end{claim}
\begin{proof}
We will prove the claim using induction. Let $P(n)$ denote the claim for a set of $n$ functions. Using $P(2)$ as a base case case, we have $f_1$ and $f_2$ as two injective functions. With $f_1 \circ f_2$ being their composition, when applying Proposition 9.7 (i), we know that $f_1 \circ f_2$ is also injective. Therefore $P(2)$ is true.

Assuming $P(n)$ is true for some $n \in N$, we will use this fact to prove it is also true for $P(n + 1)$. Ascending up to $P(n + 1)$, we have an $n + 1$ collection of functions, \[ f_1, f_2, \dotsc, f_n, f_{n + 1} \] and we are assembling the following composition \[ f_1 \circ f_2 \circ \dotsc \circ f_n \circ f_{n + 1}. \] Let $g$ be a function, representing $f_1 \circ f_2 \circ \dotsc \circ f_n$. We can rewrite our proposed composition as $g \circ f_{n + 1}.$ By hypothesis we know that $g$ is injective. From our initial assumptions, we know that $f_{n + 1}$ is also injective. Then, applying Proposition 9.7 (i) to $g \circ f_{n + 1}$ (a composition of two injective functions), we know the result will also be injective.

Thus, any composition of injective functions will also be injective, for all $n \in \bfN$, where $n \geq 2$. This concludes the induction, and the proof.
\end{proof}

\begin{claim}
For any $n \geq 2$, if $f_1, f_2, \dotsc, f_n$ are each surjective, then $f_1 \circ f_2 \circ \dotsb \circ f_n$ is surjective.
\end{claim}
\begin{proof}
We will prove the claim using induction. Let $P(n)$ denote the claim for a set of $n$ functions. Using $P(2)$ as a base case case, we have $f_1$ and $f_2$ as two surjective functions. With $f_1 \circ f_2$ being their composition, when applying Proposition 9.7 (ii), we know that $f_1 \circ f_2$ is also surjective. Therefore $P(2)$ is true.

Assuming $P(n)$ is true for some $n \in N$, we will use this fact to prove it is also true for $P(n + 1)$. Ascending up to $P(n + 1)$, we have an $n + 1$ collection of functions, \[ f_1, f_2, \dotsc, f_n, f_{n + 1} \] and we are assembling the following composition \[ f_1 \circ f_2 \circ \dotsc \circ f_n \circ f_{n + 1}. \] Let $g$ be a function, representing $f_1 \circ f_2 \circ \dotsc \circ f_n$. We can rewrite our proposed composition as $g \circ f_{n + 1}.$ By hypothesis we know that $g$ is surjective. From our initial assumptions, we know that $f_{n + 1}$ is also surjective. Then, applying Proposition 9.7 (ii) to $g \circ f_{n + 1}$ (a composition of two surjective functions), we know the result will also be surjective.

Thus, any composition of surjective functions will also be surjective, for all $n \in \bfN$, where $n \geq 2$. This concludes the induction, and the proof.
\end{proof}

\begin{claim}
For any $n \geq 2$, if $f_1, f_2, \dotsc, f_n$ are each bijective, then $f_1 \circ f_2 \circ \dotsb \circ f_n$ is bijective.
\end{claim}
\begin{proof}
We will prove the claim using induction. Let $P(n)$ denote the claim for a set of $n$ functions. Using $P(2)$ as a base case case, we have $f_1$ and $f_2$ as two bijective functions. With $f_1 \circ f_2$ being their composition, when applying Proposition 9.7 (iii), we know that $f_1 \circ f_2$ is also bijective. Therefore $P(2)$ is true.

Assuming $P(n)$ is true for some $n \in N$, we will use this fact to prove it is also true for $P(n + 1)$. Ascending up to $P(n + 1)$, we have an $n + 1$ collection of functions, \[ f_1, f_2, \dotsc, f_n, f_{n + 1} \] and we are assembling the following composition \[ f_1 \circ f_2 \circ \dotsc \circ f_n \circ f_{n + 1}. \] Let $g$ be a function, representing $f_1 \circ f_2 \circ \dotsc \circ f_n$. We can rewrite our proposed composition as $g \circ f_{n + 1}.$ By hypothesis we know that $g$ is bijective. From our initial assumptions, we know that $f_{n + 1}$ is also bijective. Then, applying Proposition 9.7 (iii) to $g \circ f_{n + 1}$ (a composition of two bijective functions), we know the result will also be bijective.

Thus, any composition of bijective functions will also be bijective, for all $n \in \bfN$, where $n \geq 2$. This concludes the induction, and the proof.
\end{proof}

\newpage
\begin{exer}
Suppose $f: A \to B$ and $g: B \to C$, so $g \circ f$ is a function $A \to C$. Prove the following claims:

If $g \circ f$ is injective, then $f$ is injective.

If $g \circ f$ is surjective, then $g$ is surjective.

% (Hint: For any $c \in C$, you want to show there’s a $b \in B$ such that $g(b) = c$. The hypothesis tells you that there exists some $a \in A$ such that $(g \circ f)(a) = c$.)

If $g \circ f$ is bijective, then $f$ is injective and $g$ is surjective. Explain why (use the previous claims) and give an example to show that $f$ doesn’t have to be surjective and $g$ doesn’t have to be injective.

% (Hint: Give an example where $f$ is a function from a set with 2 elements to a set with 3 elements, and $g$ is from the set with 3 elements back to a set with 2 elements, such as $f: \{1, 2\} \to \to {11, 12, 13\}$ and $g: \{11, 12, 13\} \to \{21, 22\}. Even with this hint, you still have to state some specific $f$ and $g$ and explain why they satisfy what the example is asking for.)
\end{exer}

\begin{claim}
If $g \circ f$ is injective, then $f$ is injective.
\end{claim}
\begin{proof}
Let $g \circ f : A \to C = g(f(a))$. We are told that $g \circ f$ is injective. We intend to show that this implies that $f$ is also injective. Suppose we have $f(a_1) = f(a_2)$ for some $a_1, a_2 \in A$. We can feed this into $g$ on each sides to see
\begin{equation}
\begin{split}
          f(a_1) &= f(a_2) \\
       g(f(a_1)) &= g(f(a_2)) \\
(g \circ f)(a_1) &= (g \circ f)(a_2).
\end{split}
\end{equation}
Because $g \circ f$ is injective, $a_1$ must be equal to $a_2$. Because $a_1$ and $a_2$ are arbitrary members of $A$, this is true for any $a_1, a_2 \in A$. Therefore $f$ must be injective. This completes the proof.
\end{proof}

\begin{claim}
If $g \circ f$ is surjective, then $g$ is surjective.
\end{claim}
\begin{proof}
Let $g \circ f : A \to C = g(f(a))$. We are told that $g \circ f$ is surjective. We intend to show that this implies $g$ is also surjective, i.e., \[ \exists c \in C, \forall b \in B \text{ such that } g(b) = c. \] Assume we have some arbitrary $c \in C$. Because $g \circ f$ is surjective, this means there's an $a \in A$ which satisfies the following equation: \[ (g \circ f)(a) = c. \] However, this is the same as writing \[ g(f(a)) = c. \] We know from the claim that $f$ maps $A
\to B$. Therefore we could say, $f(a) = b$, where $b \in B$. This means we've found a $b$ that satisfies $g(b) = c$, for any $c \in C$. Because $b$ is arbitrary, this must be true for all $b \in B$, and therefore $g$ must be surjective. This completes the proof.
\end{proof}

\begin{claim}
If $g \circ f$ is bijective, then $f$ is injective and $g$ is surjective. Explain why (use the previous claims) and give an example to show that $f$ doesn’t have to be surjective and $g$ doesn’t have to be injective.
\end{claim}
\begin{proof}
Let $g \circ f: A \to C = g(f(a))$. We are told that $g \circ f$ is bijective. We intend to show this implies $f$ is injective, and $g$ is surjective. If $g \circ f$ is bijective, this means it is both surjective and injective.

From the claim and associated proof for 4.1, we know that injectivity for $g \circ f$ is contingent on $f$'s injectivity. This is because this property of $f$ determines the domain of inputs that are fed into $g$ (i.e., $f$'s inputs are associated with unique outputs). So, $g$ could be injective or not, but it doesn't affect the resulting composition, $g \circ f$.

Secondly, from the claim and associated proof of 4.2, we know that surjectivity for $g \circ f$ is contingent upon how $g$'s surjectivity requires every element of $C$ to have a corresponding $b$ as an input to $g()$.

To show that it's not necessary for $f$ to be surjective and $g$ to be injective for $g \circ f$ to be bijective, take the following definitions for sets $A$, $B$, and $C$, and functions $f$, $g$ and $g \circ f$:
\begin{equation}
\begin{split}
                 A &= \{ 1, 2 \} \\
                 B &= \{ -2, 2, 4 \} \\
                 C &= \{ 4, 16 \} \\
        f: A \to B &= f(a) = 2a \\
        g: B \to C &= g(b) = b^2 \\
g \circ f: A \to C &= (g \circ f)(a) = g(f(a)) = (2a)^2
\end{split}
\end{equation}
Having established our sets and functions, we can evaluate the functions on their related sets:

\begin{equation}
\begin{split}
          f(a_1) &= 2(1) = 2 \\
          f(a_2) &= 2(2) = 4 \\
          g(b_1) &= (-2)^2 = 4 \\
          g(b_2) &= (2)^2 = 4 \\
          g(b_3) &= (4)^2 = 16 \\
(g \circ f)(a_1) &= (2(1))^2 = 4 \\
(g \circ f)(a_2) &= (2(2))^2 = 16
\end{split}
\end{equation}

As demonstrated above, we have verified that
\begin{itemize}
  \item $f$ is injective but not surjective (because all $a$'s map to a different $b$'s, but not all elements of $B$ have a corresponding $a$),
  \item that $g \circ f$ is bijective (both elements of $A$ map to different elements of $C$, and all elements of $A$ are mapped to elements in $C$), and
  \item that $g$ is surjective but not injective (because all $b \in B$ have at least one value of $c$ associated with them, but two elements of $B$ map onto one element of $C$).
\end{itemize}
\end{proof}

\end{document}
