\documentclass[12pt,oneside]{amsart}

\title{Math 287 Homework 9}
\author{Andrew Moore}
\date{November 13, 2021} % the due date of the homework

\usepackage[T1]{fontenc}
\usepackage{amsmath,amsfonts,amssymb,amsthm}
\usepackage[letterpaper,margin=1.5in]{geometry}
\usepackage{fancyhdr}
\pagestyle{fancy}
\usepackage{enumitem}
\usepackage{graphicx}
\graphicspath{ {../assets/} }

% Extra space between lines
\linespread{2.4}

\theoremstyle{remark}
\newtheorem{exer}{Exercise}
\newtheorem{claim}{Claim}[exer]

\newcommand{\bfN}{\mathbf{N}}
\newcommand{\bfZ}{\mathbf{Z}}
\newcommand{\bfR}{\mathbf{R}}

\newenvironment{answer}{\bigskip\noindent\emph{Answer.}}{\hfill$\diamond$\newline}

\begin{document}
\maketitle

%
%
% Question 1 (done)
%
%
\begin{exer}
Find or create a graphic to include, that you want to share.
\end{exer}
\begin{center}
% note the \graphicspath argument above
\includegraphics[width=0.75\textwidth]{pew-newspaper-revenue.png}
\end{center}

%
%
% Question 2
%
%
\newpage
\begin{exer}
Proposition 10.8. For all $x, y \in \bfR$:
\begin{itemize}
  \item[ (ii) ] $|xy| = |x| \cdot |y|$.
  \item[ (iii) ] $-|x| \leq x \leq |x|$.
\end{itemize}
\end{exer}
\begin{claim}[10.8 (ii)]
$\forall x, y \in \bfR: |xy| = |x| \cdot |y|$
\end{claim}
\begin{proof}
Let $x, y \in \bfR$. We will prove the claim using the following cases:
\begin{itemize}
  \item[1.] both $x, y > 0$.
  \item[2.] both $x, y < 0$.
  \item[3.] one of $x$ or $y$ is $<0$.
  \item[4.] one or both of $x$ and $y$ are equal to 0.
\end{itemize}

Case 1. Take an element $z \in \bfR$, which is the product of $x$ and $y$, i.e., $x \cdot y = z$. Because $x$ and $y$ are positive, we know that $z$ is also positive. Therefore, we can write
\begin{equation}
\begin{split}
|x| \cdot |y| &= x \cdot y\text{ (definition of absolute value)} \\
              &= z \\
              &= |z| \\
              &= |x \cdot y|.
\end{split}
\end{equation}

Case 2. Assume $z$ is the product of two negative real numbers denoted $-x$ and $-y$. We know from Prop. 8.18 that the product of two negative (real) numbers is a positive (real) number. Therefore, $z$ is still positive. Thus, we can write
\begin{equation}
\begin{split}
|-x| \cdot |-y| &= x \cdot y \text{ (definition of absolute value)} \\
                &= z \\
                &= |z| \\
                &= |-x \cdot -y|.
\end{split}
\end{equation}

Case 3. Assume we have two real numbers, $-x$ (negative) and $y$ (positive). From Prop. 8.22 (iii), we know that the product of a negative real number and a positive real number is a negative number. We will call this product $-z$. By the definition of absolute values, we know that $|-z| = z$. So, we can write
\begin{equation}
\begin{split}
|-x| \cdot |y| &= x \cdot y \text{ (definition of absolute value)} \\
               &= z \\
               &= |-z| \\
               &= |-x \cdot y|.
\end{split}
\end{equation}

Case 4. In an instance where either or both of $x$ and $y$ are 0, we can apply Proposition 10.8 (i), which lets us conclude $0 = |0| = |0 \cdot 0| = |x \cdot 0| = |0 \cdot y|$.

Thus we have shown that $|x| \cdot |y| = |x \cdot y|$ for all $x, y \in \bfR$. This concludes the proof.
\end{proof}

\begin{claim}[10.8 (iii)]
$\forall x \in \bfR: -|x| \leq x \leq |x|$
\end{claim}
\begin{proof}
Let $x \in \bfR$. First, in the instance where $x = 0$, we have \[ -|0| \leq 0 \leq |0| \Rightarrow 0 = 0 = 0. \] So, the statement holds. Then, consider the case when $x > 0$. We can see that \[ 0 < x = |x| \Rightarrow 0 < x \leq |x| \] by the definition of absolute value (i.e., it is true to say $x$ is less than \emph{or equal to} $|x|$).
\end{proof}

%
%
% Question 3
%
%
\newpage
\begin{exer}
Claim: $-y < x < y$ if and only if $|x| < y$.
% Hint: consider cases depending on whether x >= 0 or x < 0.
\end{exer}
\begin{proof}

\end{proof}

%
%
% Question 4
%
%
\newpage
\begin{exer}
Proposition 10.10. Let $x, y, z \in \bfR$.
\begin{itemize}
  \item[ (i) ] $|x - y| = 0$ if and only if $x = y$.
  \item[ (ii) ] $|x - y| = |y - x|$.
  \item[ (iii) ] $|x - z| \leq |x - y| + |y - z|$.
\end{itemize}
\end{exer}
\begin{claim}
$|x - y| = 0$ if and only if $x = y$.
\end{claim}
\begin{proof}
$\Rightarrow$ Let $j = (x - y)$. We  are told that $|j| = 0$. From Proposition 10.8, we know that $j$ must be equal to 0. By Axiom 8.4 and Proposition 8.11, this means that $x$ must equal $y$.

$\Leftarrow$ Assume $x = y$. Via substitution we can say
\begin{equation}
\begin{split}
&            |x - y| = 0 \\
&\Rightarrow |x - x| = 0 \\
&\Rightarrow     |0| = 0 \\
&\Rightarrow       0 = 0.
\end{split}
\end{equation}

Thus, we have shown that if $|x - y| = 0$, $x$ must equal $y$; and we have shown that if $x = y$, the absolute value of $(x - y)$ must be equal to 0. This concludes the proof.
\end{proof}

\begin{claim}
$|x - y| = |y - x|$
\end{claim}
\begin{proof}
First, consider a case where either $x = 0$ or $y = 0$. For brevity, we will focus on $x = 0$, but the same conclusion follows if we were to use $y$, because $x$ and $y$ are arbitrary real numbers. If $x = 0$, then we have
\begin{align*}
|x - y| &= |0 - y|  &  |y - x| &= |y - 0| \\
        &= |-y|     &          &= |y| \\
        &= y        &          &= y.
\end{align*}

Now consider a case where $x > y$, and let $j = x - y$ and $h = y - x$. So, $x - y \in \bfR_{>0}$. This means that $j \in \bfR_{>0}$. By the definition of absolute value, $|j| \in \bfR_{>0}$, because $|j| = j$.
\end{proof}

\begin{claim}
$|x - z| \leq |x - y| + |y - z|$
\end{claim}

\end{document}
