\documentclass[12pt,oneside]{amsart}

\title{Math 287 Homework 1}
\author{Andrew Moore}
\date{September 3, 2021} % the due date of the homework

\usepackage[T1]{fontenc}
\usepackage{amsmath,amsfonts,amssymb,amsthm}
\usepackage[letterpaper,margin=1.5in]{geometry}

% Extra space between lines
\linespread{2.4}

\theoremstyle{remark}
\newtheorem{exer}{Exercise}

\newenvironment{answer}{\bigskip\noindent\emph{Answer.}}{\hfill$\diamond$\newline}

\begin{document}
\maketitle

\newpage
\begin{exer}
Proposition 1.11(vi). If $m$, $n$, $p$, and $q$ are integers, then \newline
$(m(n+p))q = (mn)q + m(pq)$.
\end{exer}

\begin{proof}
Let $m, n, p, q \in Z.$ We will use axioms related to multiplication to show that $(m(n + p))q = (mn)q + m(pq):$

\begin{align*}
(m(n + p))q &= (m(n + p))q \\
            &= (mn + mp)q \tag{Axiom 1.1.3} \\
            &= mnq + mpq \tag{Axiom 1.1.3} \\
            &= (mn)q + m(pq). \tag{Axiom 1.1.5}
\end{align*}
\end{proof}

\newpage
\begin{exer}
Proposition 1.22(i). For all $m \in \mathbf{Z}$, $-(-m) = m$.
\end{exer}

\begin{proof}
Let $m \in Z$.

\begin{align*}
-(-m) &= -(-m) \\
      &= -1 \cdot (-1 \cdot m) \\
      &= (-1 \cdot -1) \cdot m \tag{Axiom 1.1.4} \\
      &= 1 \cdot m \tag{Corollary 1.21} \\
      &= m
\end{align*}
\end{proof}

\newpage
\begin{exer}
Proposition 1.22(ii). $-0 = 0$.
\end{exer}

\begin{proof}
We can rewrite the equation as $-1 \cdot 0 = 0$. This form now resembles what was/will be proven in Proposition 1.14:
\begin{align*}
                 -0 &= 0 \\
         -1 \cdot 0 &= 0 \\
-1 \cdot (1 + (-1)) &= 0 \tag{Replacement} \\
      -1 + (-1)(-1) &= 0 \tag{Axiom 1.1.3} \\
             -1 + 1 &= 0 \\
                  0 &= 0
\end{align*}
\end{proof}

\newpage
\begin{exer}
Proposition 1.20. In this problem, the textbook gives a proof. Your homework is to explain the proof in more detail.
\end{exer}

\begin{proof}
-enter your proof here-
\end{proof}

\newpage
\begin{exer}
Proposition 1.14.

Hint: $0 + 0 = 0.$
\end{exer}

\begin{proof}
\begin{align*}
m \cdot 0 &= m \cdot (1 + (-1)) \\
          &= m + m(-1) \\
          &= m + (-m) \\
          &= m - m \\
          &= 0
\end{align*}
\end{proof}



\end{document}
