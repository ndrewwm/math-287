\documentclass[12pt,oneside]{amsart}

\title{Math 287 Homework 1}
\author{Andrew Moore}
\date{September 3, 2021} % the due date of the homework

\usepackage[T1]{fontenc}
\usepackage{amsmath,amsfonts,amssymb,amsthm}
\usepackage[letterpaper,margin=1.5in]{geometry}

% Extra space between lines
\linespread{2.4}

\theoremstyle{remark}
\newtheorem{exer}{Exercise}

\newenvironment{answer}{\bigskip\noindent\emph{Answer.}}{\hfill$\diamond$\newline}

\begin{document}
\maketitle

\newpage
\begin{exer}
Proposition 1.11(vi). If $m$, $n$, $p$, and $q$ are integers, then \newline
$(m(n+p))q = (mn)q + m(pq)$.
\end{exer}

\begin{proof}
Let $m, n, p, q \in Z.$ We will use axioms related to multiplication to show that $(m(n + p))q = (mn)q + m(pq):$

\begin{align*}
(m(n + p))q &= (m(n + p))q \\
            &= (mn + mp)q \tag{Axiom 1.1.3} \\
            &= mnq + mpq \tag{Axiom 1.1.3} \\
            &= (mn)q + m(pq). \tag{Axiom 1.1.5}
\end{align*}
\end{proof}

\newpage
\begin{exer}
Proposition 1.22(i). For all $m \in \mathbf{Z}$, $-(-m) = m$.
\end{exer}

\begin{proof}
Let $m$ \in Z.
\end{proof}

\end{document}
