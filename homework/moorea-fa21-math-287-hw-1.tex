\documentclass[12pt,oneside]{amsart}

\title{Math 287 Homework 1}
\author{Andrew Moore}
\date{September 3, 2021} % the due date of the homework

\usepackage[T1]{fontenc}
\usepackage{amsmath,amsfonts,amssymb,amsthm}
\usepackage[letterpaper,margin=1.5in]{geometry}

% Extra space between lines
\linespread{2.4}

\theoremstyle{remark}
\newtheorem{exer}{Exercise}

\newenvironment{answer}{\bigskip\noindent\emph{Answer.}}{\hfill$\diamond$\newline}

\begin{document}
\maketitle

\newpage
\begin{exer}
Proposition 1.11(vi). If $m$, $n$, $p$, and $q$ are integers, then \newline
$(m(n+p))q = (mn)q + m(pq)$.
\end{exer}

\begin{proof}
Let $m, n, p, q \in \mathbf{Z}.$ We will use axioms related to multiplication to show that $(m(n + p))q = (mn)q + m(pq):$

\begin{align*}
(m(n + p))q &= (m(n + p))q \\
            &= (mn + mp)q \tag{Axiom 1.1.3} \\
            &= mnq + mpq \tag{Axiom 1.1.3} \\
            &= (mn)q + m(pq). \tag{Axiom 1.1.5}
\end{align*}
\end{proof}

\newpage
\begin{exer}
Proposition 1.22(i). For all $m \in \mathbf{Z}$, $-(-m) = m$.
\end{exer}

\begin{proof}
Let $m \in Z$. We can start by re-expressing $-(-m)$ as $-1 \cdot (-1 \cdot m)$. Rearranging the terms using our axioms, and simplifying, we get:
\begin{align*}
-(-m) &= -1 \cdot (-1 \cdot m) \\
      &= (-1 \cdot -1) \cdot m \tag{Axiom 1.1.4} \\
      &= 1 \cdot m \tag{Corollary 1.21} \\
      &= m. \tag{Axiom 1.3}
\end{align*}
It must be for any integer $m$, $-(-m) = m$.
\end{proof}

\newpage
\begin{exer}
Proposition 1.22(ii). $-0 = 0$.
\end{exer}

\begin{proof}
We can rewrite the equation as $0 = -1 \cdot 0$. Also observe, as defined in Axiom 1.4, that $1 + (-1) = 0$. With this in hand, we can see that
\begin{align*}
0 &= -0 \\
  &= -1 \cdot 0 \\
  &= -1 \cdot (1 + (-1)) \tag{Replacement} \\
  &= -1 + (-1)(-1) \tag{Axiom 1.1.3} \\
  &= -1 + 1 \tag{Simplifying, using Corollary 1.21} \\
  &= 0.
\end{align*}
\end{proof}

\newpage
\begin{exer}
% Proposition 1.20. In this problem, the textbook gives a proof. Your homework is to explain the proof in more detail.

Proposition 1.20. For all $m,n \in \mathbf{Z}, (-m)(-n) = mn$.
\end{exer}

\begin{proof}
Let $m,n \in \mathbf{Z}$. By Axiom 1.4, $m + (-m) = 0$ and $n + (-n) = 0$. Multiplying both sides of the first equation (on the right) by $n$ and the second (on the left) by $-m$ gives, after applying Proposition 1.14 on the right-hand sides,

$(m + (-m))n = 0$ and $(-m)(n + (-n)) = 0$.

With Axiom 1.1(iii) and Proposition 1.6 we deduce

$mn + (-m)n = 0$ and $(-m)n + (-m)(-n) = 0$.

It remains to use Axiom 1.1(i) on the left and then Proposition 1.10 to conclude

$mn = (-m)(-n)$.
\end{proof}

\newpage
\begin{exer}
Proposition 1.14. For all $m \in \mathbf{Z}, m \cdot 0 = 0 = 0 \cdot m$.

Hint: $0 + 0 = 0.$
\end{exer}

\begin{proof}
Observe that, by Axiom 1.4: $1 + (-1) = 0$. Replacing this fact into the equation, and distributing we get
\begin{align*}
m \cdot 0 &= m \cdot (1 + (-1)) \\
          &= m + m(-1) \tag{Axiom 1.1.3} \\
          &= m + (-m) \tag{Axiom 1.1.3} \\
          &= 0. \tag{Axiom 1.4}
\end{align*}
It must be that for any integer $m, m \cdot 0 = 0$.
\end{proof}

\end{document}
