\documentclass[12pt,oneside]{amsart}

\title{Math 287 Homework 11}
\author{Andrew Moore}
\date{December 5, 2021} % the due date of the homework

\usepackage[T1]{fontenc}
\usepackage{amsmath,amsfonts,amssymb,amsthm}
\usepackage[letterpaper,margin=1.5in]{geometry}
\usepackage{fancyhdr}
\usepackage{tikz}
\pagestyle{fancy}
\usepackage{enumitem}

% Extra space between lines
\linespread{2.4}

\theoremstyle{remark}
\newtheorem{exer}{Exercise}
\newtheorem{claim}{Claim}[exer]

\newcommand{\bfN}{\mathbf{N}}
\newcommand{\bfZ}{\mathbf{Z}}
\newcommand{\bfR}{\mathbf{R}}

\newenvironment{answer}{\bigskip\noindent\emph{Answer.}}{\hfill$\diamond$\newline}

\begin{document}
\maketitle

\begin{exer}
Suppose $f: A \to B$ is surjective. Then, for any set C and functions $g_1, g_2 : B \to C$, if $g_1 \circ f = g_2 \circ f$, $g_1 = g_2$.
\end{exer}

\begin{proof}
Let $f: A \to B$ and $g_1, g_2 : B \to C$ be functions. We are examining compositions between $f$ and $g_1, g_2$. From our hypothesis, we have $g_1 \circ f = g_2 \circ f$. From the definition of a composition, we know $g_1 \circ f: A \to C$ is defined by $(g_1 \circ f)(a) = c$ for all $a \in A$, and $g_2 \circ f: A \to C$ is defined by $(g_2 \circ f)(a) = c$ for all $a \in A$. Additionally, we know that $f$ is surjective. This means that for every $b \in B$, there exists some $a \in A$ such that $f(a) = b$. Thus, we can write
\begin{align*}
     g_1 \circ f &= g_2 \circ f \tag{starting assumption} \\
(g_1 \circ f)(a) &= (g_2 \circ f)(a) \\
       g_1(f(a)) &= g_2(f(a)) \tag{definition of a composition} \\
          g_1(b) &= g_2(b) \\
             g_1 &= g_2.
\end{align*}
Because for all $b \in B, \exists a \in A$ such that $f(a) = b$ (surjectivity of $f$), and $g_1 \circ f = g_2 \circ f$, we know that the $a$ being fed to $f$ is the same element of $A$. This means the result of $f(a)$, is the same $b$ on both sides of the equality. Thus we have shown $g_1 = g_2$. This concludes the proof.
\end{proof}

\newpage
\begin{exer}
a. Find and prove a formula for $2 + 5 + 8 + 11 + \cdots + (3n - 1)$.

% Your formula will involve n^2 but nothing higher than that. For homework, give a clear statement, like: “Claim: 2 + 5 + 8 + ... + (3n - 1) = (a formula)”, and then prove it. You do not need to write the steps you took to find the formula.

b. Prove: for all positive odd integers $n, 5^n - n^2$ is divisible by 4.

% Positive odd integers means n = 1, 3, 5, .... You can either write an induction that takes steps of size 2, or else set n = 2k + 1, so n = 0, 1, 2, ..., and use “normal” induction with k.
\end{exer}

\begin{claim}
I claim that \[ 2 + 5 + 8 + 11 + \cdots + (3n - 1) = \sum_{i = 1}^n (3n - 1) = \frac{1}{2}(3n^2 + n). \]
\end{claim}

\begin{proof}
We will show that $\sum_{i = 1}^n (3i - 1) = \frac{1}{2}(3n^2 + n)$, for $n \geq 1$ using induction. As a base case, $n = 1$, we have \[ \sum_{i = 1}^1 (3i - 1) = 2 = \frac{1}{2}(3(1)^2 + 1). \]

We will now assume the formula holds for all natural numbers up to $n$. We will then use this to show that it also holds for $n + 1$. That is, we intend to demonstrate $\sum_{i = 1}^{n+1} (3i - 1) = \frac{1}{2}(3(n + 1)^2 + (n + 1))$. Re-expressing the sum, we see
\begin{align*}
\sum_{i = 1}^{n+1} (3i - 1) &= \sum_{i = 1}^{n} (3i - 1) + (3n + 2) \\
                            &= \frac{1}{2}(3n^2 + n) + (3n + 2) \tag{by hypothesis} \\
                            &= \frac{(3n^2 + n)}{2} + \frac{2(3n + 2)}{2} \tag{ensuring a common denominator} \\
                            &= \frac{(3n^2 + n) + (6n + 4)}{2} \\
                            &= \frac{3n^2 + 7n + 4}{2} \tag{simplifying} \\
                            &= \frac{1}{2}(3n^2 + 7n + 4) \\
                            &= \frac{1}{2}(3(n^2 + 2n + 1) + (n + 1)) \\
                            &= \frac{1}{2}(3(n + 1)^2 + (n + 1)).
\end{align*}
This concludes the induction, and the proof.
\end{proof}

\begin{claim}
For all positive odd integers $n, 5^n - n^2$ is divisible by 4.
\end{claim}

\begin{proof}
By induction on $n$. As a base case, $n = 1$, we see $5^1 - 1^2 = 4$. Assuming the statement holds for all odd integers up to $n$, we will show that it also holds for $n + 2$. To state that $5^n - n^2$ is divisible by 4 is to assert $\exists y \in \bfZ$ such that $4y = 5^n - n^2$. We will demonstrate that there also $\exists z \in \bfZ$ such that $4z = 5^{n + 2} - (n + 2)^2$. First, note that $5^n = 4y + n^2$. Then, we can see
\begin{align*}
4z &= 5^{n + 2} - (n + 2)^2 \\
4z &= 5^n \cdot 5^2 - (n + 2)^2 \\
4z &= (4y + n^2) \cdot 5^2 - (n + 2)^2 \tag{by hypothesis} \\
4z &= 100y + 25n^2 - (n^2 + 4n + 4) \tag{distributing and expanding} \\
4z &= 100y + 24n^2 - 4n - 4 \tag{simplifying} \\
4z &= 4(25y + 6n^2 - n - 1) \\
 z &= 25y + 6n^2 - n - 1.
\end{align*}
We have shown that there exists an integer $z$, such that $4z = 5^{n + 2} - (n + 2)^2$, meaning that the result of the statement is divisible by 4. We can thus conclude the statement is true for all odd integers. This concludes the induction and the proof.
\end{proof}

\newpage
\begin{exer}
Proposition 11.25. Let $b, c, p, q \in \bfR$. If $x^2 - bx - c = 0$ has two solutions $s$ and $t$, and if we define a sequence $(a_k)_{k = 1}^\infty$ by $a_k := ps^k + qt^k$, then this sequence satisfies a recurrence relation $a_n = ba_{n - 1} + ca_{n - 2}$ for all $n \geq 3$.
\end{exer}

% This problem is less complicated than it might at first appear. They give you a formula for a_k; plug that into the recurrence relation, and check that both sides come out to the same value, once you do some algebraic simplifications (group like terms and pull out common factors).

\begin{proof}
By induction on $n$. Let $b, c, p, q \in \bfR$. We will show that the sequence $a_k$ satisfies the recurrence relation $a_n$ for all $n \geq 3$. As a base case, n = 3, we have
\begin{align*}
a_n &= ba_{3 - 1} + ca_{3 - 2} \\
    &= ba_2 + ca_1 \\
    &= b(ps^2 + qt^2) + c(ps + qt) \\
    &= (b + c)(p^2s^3 + q^2t^3) \\
    &= ... ... ...
\end{align*}
\end{proof}

\newpage
\begin{exer}
Re-do a problem: Homework 5, problem 4: "$(A \times B) \cap (C \times D) = (A \cap C) \times (B \cap D).$"
\end{exer}

\begin{answer}
I was marked down for not being adequately thorough in my answer. Here is my revised proof.
\begin{proof}
Take an element $x$ on the left-hand side, and let $x = (i, j)$. This element is in the intersection of $A \times B$ and $C \times D$, that is, $x \in A \times B$ \emph{and} $x \in C \times D$. This means that $i \in A$ and $j \in B$ and that $i \in C$ and $j \in D$. From these observations, we can state that $i \in A \cap C$ and $j \in B \cap D$. So, we can conclude that $x = (i, j) \in (A \cap C) \times (B \cap D)$, i.e., \[ (A \times B) \cap (C \times D) \subseteq (A \cap C) \times (B \cap D). \]

Now, we will examine an element $y$ on the right-hand side, letting $y = (k, l)$. The element $y$ is in the cross product of the two intersections of $A \cap C$ and $B \cap D$. By definition, this means that $k \in A \cap C$ and $l \in B \cap D$. Stated explicitly, this means $k \in A$, $l \in B$, $k \in C$, and $l \in D$.
If we take the cross product of $A$ and $B$, $(k, l)$ is a member of $A \times B$. Similarly, if we take the cross product of $C$ and $D$, $(k, l)$ is a member of $C \times D$. Thus, $y = (k, l) \in (A \times B) \cap (C \times D)$. So, we can say \[ (A \times B) \cap (C \times D) \supseteq (A \cap C) \times (B \cap D). \]

We have shown that \[ (A \times B) \cap (C \times D) \subseteq (A \cap C) \times (B \cap D) \] and \[ (A \times B) \cap (C \times D) \supseteq (A \cap C) \times (B \cap D). \] Thus, we can conclude \[ (A \times B) \cap (C \times D) = (A \cap C) \times (B \cap D). \]
\end{proof}

Unlike other areas of this course, I've felt more immediately comfortable with set theory, and I have enjoyed its appearances across the different classes I've taken thus far. From what I've been exposed to, it seems like proofs within set theory can often be written as direct proofs, or proof by contrapositive. The rules regarding relations between elements/sets feel very logical, and this results in proofs that feel more transparent when reading them (in my opinion). I feel like I'm more likely to grok a proof about sets on my first 1-2 readings than with other topics.

For this problem, I tried to focus on the feedback from the grader. They indicated I had been too sparse in my setup of the elements I would track on either side of the proposed equality. Precise language is always important for writing in mathematics, but when discussing sets, it feels like you have an ability (and thus, an obligation) to provide clear descriptions for the objects you're discussing.
\end{answer}

\end{document}
