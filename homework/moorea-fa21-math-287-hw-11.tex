\documentclass[12pt,oneside]{amsart}

\title{Math 287 Homework 11}
\author{Andrew Moore}
\date{December 5, 2021} % the due date of the homework

\usepackage[T1]{fontenc}
\usepackage{amsmath,amsfonts,amssymb,amsthm}
\usepackage[letterpaper,margin=1.5in]{geometry}
\usepackage{fancyhdr}
\usepackage{tikz}
\pagestyle{fancy}
\usepackage{enumitem}

% Extra space between lines
\linespread{2.4}

\theoremstyle{remark}
\newtheorem{exer}{Exercise}
\newtheorem{claim}{Claim}[exer]

\newcommand{\bfN}{\mathbf{N}}
\newcommand{\bfZ}{\mathbf{Z}}


\newenvironment{answer}{\bigskip\noindent\emph{Answer.}}{\hfill$\diamond$\newline}

\begin{document}
\maketitle

\begin{exer}
Suppose $f: A \to B$ is surjective. Then, for any set C and functions $g_1, g_2 : B \to C$, if $g_1 \circ f = g_2 \circ f$, $g_1 = g_2$.
\end{exer}

\begin{proof}
Let $f: A \to B$ and $g_1, g_2 : B \to C$ be functions. We are examining compositions between $f$ and $g_1, g_2$. From our hypothesis, we have $g_1 \circ f = g_2 \circ f$. From the definition of a composition, we know $g_1 \circ f: A \to C$ is defined by $(g_1 \circ f)(a) = c$ for all $a \in A$, and $g_2 \circ f: A \to C$ is defined by $(g_2 \circ f)(a) = c$ for all $a \in A$. Additionally, we know that $f$ is surjective. This means that for every $b \in B$, there exists some $a \in A$ such that $f(a) = b$. Thus, we can write
\begin{align*}
     g_1 \circ f &= g_2 \circ f \tag{starting assumption} \\
(g_1 \circ f)(a) &= (g_2 \circ f)(a) \\
       g_1(f(a)) &= g_2(f(a)) \tag{definition of a composition} \\
          g_1(b) &= g_2(b) \\
             g_1 &= g_2.
\end{align*}
Because for all $b \in B, \exists a \in A$ such that $f(a) = b$ (surjectivity of $f$), and $g_1 \circ f = g_2 \circ f$, we know that the $a$ being fed to $f$ is the same element of $A$. This means the result of $f(a)$, is the same $b$ on both sides of the equality. Thus we have shown $g_1 = g_2$. This concludes the proof.
\end{proof}

\newpage
\begin{exer}
a. Find and prove a formula for $2 + 5 + 8 + 11 + \cdots + (3n - 1)$.

b. Prove: for all positive odd integers $n, 5^n - n^2$ is divisible by 4.
\end{exer}

\newpage
\begin{exer}
Proposition 11.25.
\end{exer}

\newpage
\begin{exer}
Re-do a problem.
\end{exer}

\end{document}
