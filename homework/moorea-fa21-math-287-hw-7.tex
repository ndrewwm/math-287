\documentclass[12pt,oneside]{amsart}

\title{Math 287 Homework 7}
\author{Andrew Moore}
\date{October 29, 2021} % the due date of the homework

\usepackage[T1]{fontenc}
\usepackage{amsmath,amsfonts,amssymb,amsthm}
\usepackage[letterpaper,margin=1.5in]{geometry}
\usepackage{fancyhdr}
\pagestyle{fancy}
\usepackage{enumitem}

% Extra space between lines
\linespread{2.4}

\theoremstyle{remark}
\newtheorem{exer}{Exercise}
\newtheorem{claim}{Claim}[exer]

\newcommand{\bfN}{\mathbf{N}}
\newcommand{\bfZ}{\mathbf{Z}}
\newcommand{\bfR}{\mathbf{R}}

\newenvironment{answer}{\bigskip\noindent\emph{Answer.}}{\hfill$\diamond$\newline}

\begin{document}
\maketitle

\begin{exer}
Type in the piecewise definition of the Fibonacci numbers.
% f_n = {0, if n = 0; 1, if n = 1; f_{n - 1} + f_{n - 2}, otherwise }
\end{exer}

\newpage
\begin{exer}
Proposition 8.18: For all $m,n \in \bfR, (-m)(-n) = mn$.
\end{exer}

\newpage
\begin{exer}
Proposition 8.32: For all $x,y,z,w \in \bfR$:
\begin{enumerate}
  \item If $x < y$ then $x + z < y + z$.
  \item If $x < y$ and $z < w$ then $x + z < y + w$.
  \item If $0 < x < y$ and $0 < z \leq w$ then $xz < yw$.
  \item If $x < y$ and $z < 0$ then $yz < xz$.
\end{enumerate}

\end{exer}

\begin{claim}
If $x < y$ then...
\end{claim}
\begin{proof}
(Your proof)
\end{proof}

\begin{claim}
If $x < y$ and $z < w$ then...
\end{claim}
\begin{proof}
(Your proof)
\end{proof}

\newpage
\begin{exer}
Proposition 8.53: Every nonempty subset of $\bfR$ that is bounded below has a greatest lower bound. % Hint: If $M$ is a lower bound for $A$, then define $B = \{-x | x \in A\}$, show that $-M$ is an upper bound for $B$, and use Axiom 8.52.
\end{exer}

\end{document}
