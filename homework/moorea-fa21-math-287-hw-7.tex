\documentclass[12pt,oneside]{amsart}

\title{Math 287 Homework 7}
\author{Andrew Moore}
\date{October 29, 2021} % the due date of the homework

\usepackage[T1]{fontenc}
\usepackage{amsmath,amsfonts,amssymb,amsthm}
\usepackage[letterpaper,margin=1.5in]{geometry}
\usepackage{fancyhdr}
\pagestyle{fancy}
\usepackage{enumitem}

% Extra space between lines
\linespread{2.4}

\theoremstyle{remark}
\newtheorem{exer}{Exercise}
\newtheorem{claim}{Claim}[exer]

\newcommand{\bfN}{\mathbf{N}}
\newcommand{\bfZ}{\mathbf{Z}}
\newcommand{\bfR}{\mathbf{R}}

\newenvironment{answer}{\bigskip\noindent\emph{Answer.}}{\hfill$\diamond$\newline}

\begin{document}
\maketitle

%
%
% Question 1 (done)
%
%
\begin{exer}
Type in the piecewise definition of the Fibonacci numbers.
% f_n = {0, if n = 0; 1, if n = 1; f_{n - 1} + f_{n - 2}, otherwise }
\end{exer}

\begin{answer}
$$
f_n =
  \begin{cases}
    0,                    & \text{if n = 0} \\
    1,                    & \text{if n = 1} \\
    f_{n - 1} + f_{n - 2} & \text{otherwise}
  \end{cases}
$$
\end{answer}

%
%
% Question 2 (done)
%
%
\newpage
\begin{exer}
Proposition 8.18: For all $m,n \in \bfR, (-m)(-n) = mn$.
\end{exer}

\begin{proof}
Let $m,n \in \bfR$. First, we should note that $-m + m = 0$ and $-n + n = 0$ (Axiom 8.4). Multiplying the first equation by $n$ and the second equation by $-m$ gives us \[ n(-m + m) = 0 \text{ and } (-m)(-n + n) = 0. \] From Prop. 8.15, we know that $-m \cdot 0 = 0$ and $n \cdot 0 = 0$, so the right-hand sides of each equation remain the same. After distributing and rearranging (Prop. 8.8 and 8.1 (iii)), we have \[ mn + (-m)n = 0 \text{ and } (-m)n + (-m)(-n) = 0. \] Using Axiom 8.1 (i) to rearrange the equation on the left, we can show
\begin{align*}
&(-m)n + mn = 0 \\
&(-m)n + (-m)(-n) = 0.
\end{align*}
From Prop. 8.11, we know that the additive inverses of real numbers are unique. Therefore, $(-m)(-n)$ must be $mn$, i.e., $(-m)(-n) = mn$.
\end{proof}

%
%
% Question 3
%
%
\newpage
\begin{exer}
Proposition 8.32: For all $w,x,y,z \in \bfR$:
\begin{enumerate}
  \item If $x < y$ then $x + z < y + z$.
  \item If $x < y$ and $z < w$ then $x + z < y + w$.
  \item If $0 < x < y$ and $0 < z \leq w$ then $xz < yw$.
  \item If $x < y$ and $z < 0$ then $yz < xz$.
\end{enumerate}
\end{exer}

\begin{claim}
If $x < y$ then $x + z < y + z$.
\end{claim}
\begin{proof}
Let $x,y \in \bfR$. We are told that $x < y$, that is, $y - x \in \bfR_{>0}$. Then, suppose we take any arbitrary element $z \in \bfR$, and add $z$ to both $x$ and $y$:
\begin{equation}
\begin{split}
x + z < y + z &\Rightarrow (y + z) - (x + z) \in \bfR_{>0} \text{ (definition of <)} \\
              &\Rightarrow y + z - x - z \in \bfR_{>0} \text{ (distributing)} \\
              &\Rightarrow y - x + 0 \in \bfR_{>0} \text{ (Axiom 8.4)} \\
              &\Rightarrow x < y \text{ (definition of <).}
\end{split}
\end{equation}
We have shown that when adding an arbitrary constant value $z$ to both sides of the inequality, the quantity $z$ is zeroed out, preserving the inequality between $x$ and $y$. Thus we have shown that for all $z \in \bfR$, if $x < y$, $x + z < y + z$.
\end{proof}

\begin{claim}
If $x < y$ and $z < w$ then $x + z < y + w$.
\end{claim}
\begin{proof}
Let $w,x,y,z \in \bfR$. We are told that $x < y$ and $z < w$. That is, $y - x \in \bfR_{>0}$ and $w - z \in \bfR_{>0}$, by the definition of <. Adding $(y - x)$ and $(w - z)$ gives
\begin{equation}
\begin{split}
(y - x) + (w - z) \in \bfR_{>0} &\Rightarrow y + w - x - z \in \bfR_{>0} \\
                                &\Rightarrow (y + w) - (x) - (z) \in \bfR_{>0} \text{ (Prop. 8.22)} \\
                                &\Rightarrow (y + w) - (x + z) \in \bfR_{>0} \text{ (Axiom 8.1 (ii))} \\
                                &\Rightarrow x + z < y + w.
\end{split}
\end{equation}
\end{proof}

\begin{claim}
If $0 < x < y$ and $0 < z \leq w$ then $xz < yw$.
\end{claim}
\begin{proof}
Let $w,x,y,z \in \bfR$. We know that $y - x \in \bfR_{>0}$, $x - 0 \in \bfR_{>0}$, and $z - 0 \in \bfR_{>0}$. We also know that $w - z \in \bfR_{>0}$ or $w = z$. If $w = z$ then, we have
\begin{equation}
\begin{split}
z(y - x) \in \bfR_{>0} &\Rightarrow yz - xz \in \bfR_{>0} \\
                       &\Rightarrow xz < yz \\
                       &\Rightarrow (xz < yz) \equiv (xw < yw) \equiv (xw < yz) \equiv (xz < yw).
\end{split}
\end{equation}

If $z < w$, then $w - z \in \bfR_{>0}$. Multiplying $(y - x)$ and $(w - z)$, we have
\[ (y - x) \cdot (w - z) \in \bfR_{>0} \Rightarrow yw - yz - xw - xz \in \bfR_{>0}. \]

Letting $j = -(yz + xw + xz)$, we can see that $yw - j \in \bfR_{>0} \Rightarrow j < yw$. Given that $xz$ is part of $j$'s sum, and each of $x,w,y,z > 0$, we can conclude that $yw - xz \in \bfR_{>0}$, i.e., $xz < yw$.
\end{proof}

\begin{claim}
If $x < y$ and $z < 0$ then $yz < xz$.
\end{claim}
\begin{proof}
Let $x,y,z \in \bfR$. We are told that $x < y$ and $z < 0$. This means that $y - x \in \bfR_{>0}$ and $0 - z \in \bfR_{>0}$ (i.e., $-z \in \bfR_{>0}$). We know that $\bfR$ is closed under multiplication, so we can multiply $-z$ and $y - x$ to show that
\begin{equation}
\begin{split}
(-z) \cdot (y - x) \in \bfR_{>0} &\Rightarrow -yz \cdot (-z)(-x) \in \bfR_{>0} \\
                                 &\Rightarrow xz - yz \in \bfR_{>0} \text{ (Prop. 8.18 \& Axiom 8.1 (i))} \\
                                 &\Rightarrow yz < xz \text{ (by the definition of <).}
\end{split}
\end{equation}
\end{proof}

%
%
% Question 4
%
%
\newpage
\begin{exer}
Proposition 8.53: Every nonempty subset of $\bfR$ that is bounded below has a greatest lower bound. % Hint: If $M$ is a lower bound for $A$, then define $B = \{-x | x \in A\}$, show that $-M$ is an upper bound for $B$, and use Axiom 8.52.
\end{exer}

\begin{proof}
Let $A \subseteq \bfR$ where $A \neq \emptyset$. Let us also assume that $A$ is bounded below. We intend to show that $A$ has a greatest lower bound. That is we will prove that $\exists c \in \bfR$ such that
\begin{enumerate}
  \item $\forall a \in A, c \leq a$ and
  \item $\forall r \in \bfR$, if $r$ is a lower bound of $A$, $r \leq c$.
\end{enumerate}
First, let $M$ be a lower bound for $A$, and let $B = \{-a | a \in A\}$. If $M$ is a lower bound for $A$, this means that $\forall a \in A, M \leq a$. Then, let $b$ be an arbitrary member of $B$:
\begin{equation}
\begin{split}
             b &\in B \\
\Rightarrow -b &\in A \text{ (negation of b is in A)} \\
\Rightarrow -b &\geq M \text{ (all elements of A are bounded below by M)} \\
\Rightarrow b &\leq -M.
\end{split}
\end{equation}
Because $b$ is arbitrary, this means that $-M$ is an upper bound of B. By Axiom 8.52, we know that every non-empty subset of $\bfR$ has a least upper bound. $B$ is a non-empty subset of $\bfR$, so $\exists k \in \bfR$ such that:
\begin{enumerate}
  \item $\forall b \in B, b \leq k$ and
  \item $\forall r \in \bfR$, if $r$ is an upper bound of $B$, $k \leq r$.
\end{enumerate}
Given that we know some $k$ exists, we can say
\begin{equation}
\begin{split}
k &\in \bfR \\
\forall b \in B, b &\leq k \text{ (definition of an upper bound)} \\
\forall b \in B, -b &\geq -k \text{ (multiplying both sides by -1)} \\
\forall a \in A, a &\geq -k \text{ (because an arbitrary $a = -b$)}.
\end{split}
\end{equation}
This means that $-k$ is a lower bound of $A$. Given $k$'s relationship with $B$, we can infer that $\forall r \in \bfR$, if $r$ is a lower bound of $A$, then $-k \geq r$. This meets the criteria we established at the beginning of the proof, meaning that $-k$ must be equal to $c$, the greatest lower bound of $A$.
\end{proof}

\end{document}
