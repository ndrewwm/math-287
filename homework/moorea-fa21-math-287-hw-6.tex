\documentclass[12pt,oneside]{amsart}

\title{Math 287 Homework 6}
\author{Andrew Moore}
\date{October 8, 2021} % the due date of the homework

\usepackage[T1]{fontenc}
\usepackage{amsmath,amsfonts,amssymb,amsthm}
\usepackage[letterpaper,margin=1.5in]{geometry}
\usepackage{fancyhdr}
\pagestyle{fancy}
\usepackage{enumitem}

% Extra space between lines
\linespread{2.4}

\theoremstyle{remark}
\newtheorem{exer}{Exercise}
\newtheorem{claim}{Claim}[exer]

\newcommand{\bfN}{\mathbf{N}}
\newcommand{\bfZ}{\mathbf{Z}}


\newenvironment{answer}{\bigskip\noindent\emph{Answer.}}{\hfill$\diamond$\newline}

\begin{document}
\maketitle

%
%
% Question 1
%
%
\newpage
\begin{exer}
The derivative of $x^2$ is $2x$.
\end{exer}
\begin{proof}
Let $f(x) = x^2$.
Using the limit definition of derivative, we have
\begin{equation}
\begin{split}
  \lim_{h \to 0} \frac{f(x+h) - f(x)}{h} &= ? \\
    &= ? \\
    &= ? \\
    &= ? \\
    &= ? \\
    &= ? \\
    &= ? \\
    &= ? \\
    &= ? \\
    &= ?
\end{split}
\end{equation}
\end{proof}

%
%
% Question 2
%
%
\newpage
\begin{exer}
Project 6.9(i), transitive. Prove that the relation defined in the book is transitive.
%
% There is one step in the middle of the proof where you will use the fact that the set is , meaning that we are looking at pairs  where . When you get to that step, in other words when you use the fact that , please highlight that and state it clearly. Say why it matters that .
%
% (Hint: It’s because there’s a step where we divide by , but we can’t divide by . So, knowing  is important. Please show which step that is, and explain it in your own words.)
%
% LaTeX: The symbol that the book uses for the relation, , is typed \sim (short for “similar”).
\end{exer}

%
%
% Question 3
%
%
\newpage
\begin{exer}
Proposition 6.17.
\end{exer}

%
%
% Question 4
%
%
\newpage
\begin{exer}
Explain the proof of Proposition 6.29(i). The textbook gives a proof of Proposition 6.29(i). Rewrite the proof in more detail and with more explanation.
\end{exer}

\end{document}
