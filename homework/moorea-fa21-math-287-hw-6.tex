\documentclass[12pt,oneside]{amsart}

\title{Math 287 Homework 6}
\author{Andrew Moore}
\date{October 8, 2021} % the due date of the homework

\usepackage[T1]{fontenc}
\usepackage{amsmath,amsfonts,amssymb,amsthm}
\usepackage[letterpaper,margin=1.5in]{geometry}
\usepackage{fancyhdr}
\pagestyle{fancy}
\usepackage{enumitem}

% Extra space between lines
\linespread{2.4}

\theoremstyle{remark}
\newtheorem{exer}{Exercise}
\newtheorem{claim}{Claim}[exer]

\newcommand{\bfN}{\mathbf{N}}
\newcommand{\bfZ}{\mathbf{Z}}


\newenvironment{answer}{\bigskip\noindent\emph{Answer.}}{\hfill$\diamond$\newline}

\begin{document}
\maketitle

%
%
% Question 1 (done)
%
%
\newpage
\begin{exer}
The derivative of $x^2$ is $2x$.
\end{exer}
\begin{proof}
Let $f(x) = x^2$.
Using the limit definition of derivative, we have
\begin{equation}
\begin{split}
  \lim_{h \to 0} \frac{f(x+h) - f(x)}{h} &= \lim_{h \to 0} \frac{(x + h)^2 - x^2}{h} \\
                                         &= \lim_{h \to 0} \frac{x^2 - x^2 + 2xh + h^2}{h} \\
                                         &= \lim_{h \to 0} \frac{2xh + h^2}{h} \\
                                         &= \lim_{h \to 0} \frac{2xh}{h} + \frac{h^2}{h} \\
                                         &= \lim_{h \to 0} 2x + h \\
                                         &= 2x + 0 \\
                                         &= 2x.
\end{split}
\end{equation}
\end{proof}

%
%
% Question 2
%
%
\newpage
\begin{exer}
Project 6.9. On $\bfZ \times (\bfZ - \{0\})$ we define the relation $(m_1, n_1) \sim (m_2, n_2)$ if $m_1n_2 = n_1m_2$. Prove that the relation defined in the book is transitive.
% There is one step in the middle of the proof where you will use the fact that the set is Z x (Z - {0}), meaning that we are looking at pairs (m, n) where n != 0. When you get to that step, in other words when you use the fact that n != 0, please highlight that and state it clearly. Say why it matters that n != 0.
%
% (Hint: It’s because there’s a step where we divide by n, but we can’t divide by 0. So, knowing n != 0 is important. Please show which step that is, and explain it in your own words.)
%
% LaTeX: The symbol that the book uses for the relation, ~, is typed \sim (short for “similar”).
\end{exer}

\begin{proof}
% A relation, $\sim$, is considered \emph{transitive} if for some arbitrary $a, b, c$ within $A \times
\end{proof}

%
%
% Question 3 (done?)
%
%
\newpage
\begin{exer}
Prop. 6.17. Let $m \in \bfZ$. This number $m$ is even, iff $m^2$ is even.
\end{exer}

% P(k) = m is even
% P(q) = m^2 is even
% k <=> q === (p => q) & (-p => -q)
% k <=> q === (p => q) & (q => p)
%
% p   q  p -> q -p -> -q  q -> p
% T   F    F       T        T
% T   T    T       T        T
% F   F    T       T        T
% F   T    T       F        F
\begin{proof}
Assume that $m$ is even, i.e. $2 \mid m$. This means that $m = 2n$ for some $n \in \bfZ$. So, by the definition of powers we can write \[ m^2 = (2n)^2 = 4n^2 = 2 \cdot (2n^2). \] Because $n$ is an integer, the term $2n^2$ is also an integer, and since it is being multiplied by 2, we know the product is even.

Conversely, assume that $m$ is not even. This means that $m$ is odd, and we can write $m = 2q + 1$ for some $q \in \bfZ$. Again, by the definition of powers we have \[ m^2 = (2q + 1)^2 = 4q^2 + 4q + 1 = 2(2q^2 + 2q) + 1. \]

Let $z = (2q^2 + 2q)$. We know that the integers are closed under multiplication, and thus the product of $2z$ is also an integer. Therefore, we have \[ m^2 = 2z +1, \] which we know must be odd (Proposition 6.15).

We have shown that if $m$ is even, $m^2$ must also be even. Additionally, we have shown that if $m$ is odd, $m^2$ must also be odd. This means that $m$ is even if and only if $m^2$ is even.
\end{proof}

%
%
% Question 4
%
%
\newpage
\begin{exer}
Explain the proof of Proposition 6.29(i). The textbook gives a proof of Proposition 6.29(i). Rewrite the proof in more detail and with more explanation.
\end{exer}
% Let  m,n \in Z. (i) gcd(m,n) divides both m and n.

\begin{answer}
\end{answer}

\end{document}
