\documentclass[12pt,oneside]{amsart}

\title{Math 287 Homework 5}
\author{Andrew Moore}
\date{October 1, 2021} % the due date of the homework

\usepackage[T1]{fontenc}
\usepackage{amsmath,amsfonts,amssymb,amsthm}
\usepackage[letterpaper,margin=1.5in]{geometry}
\usepackage{fancyhdr}
\pagestyle{fancy}
\usepackage{enumitem}

% Extra space between lines
\linespread{2.4}

\theoremstyle{remark}
\newtheorem{exer}{Exercise}
\newtheorem{claim}{Claim}[exer]

\newcommand{\bfN}{\mathbf{N}}
\newcommand{\bfZ}{\mathbf{Z}}


\newenvironment{answer}{\bigskip\noindent\emph{Answer.}}{\hfill$\diamond$\newline}

\begin{document}
\maketitle

%
%
% Question 1 (done)
%
%
\newpage
\begin{exer}
Show some of your favorite equations in \emph{inline} and \textbf{display} mathematics.
\end{exer}

\begin{answer}
Some people like small equations such as $e^{\pi i}+1 = 0$
or \( a^2+b^2 = c^2 \).

My favorite small equation is $A^TA\textbf{x} = A^T\textbf{b}$.

Other people like big equations like $$ x = \frac{-b \pm \sqrt{b^2-4ac}}{2a} $$
and \[ \int x^n \, dx = \frac{1}{n+1} x^{n+1} + C . \]

My favorite big equation is
$$
\lim_{x \to 0} \frac{sin(x)}{x} = 1
$$
\end{answer}

%
%
% Question 2 (done)
%
%
\newpage
\begin{exer}
Project 5.16(ii). Prove or give a counterexample: For all sets $A, B, C$, $A \cap (B - C) = (A \cap B) - (A \cap C)$.
\end{exer}

\begin{answer}
I claim that this statement is true.

\begin{proof}
Taking an element on the left-hand side, called $x$, we can see that
\begin{align*}
&x \in A \tag{because $x \in A \cap (B - C)$} \\
&\Rightarrow  x \in (B - C) \tag{definition of $\cap$} \\
&\Rightarrow (x \in B) \wedge (x \notin C) \tag{definition of $-$} \\
&\Rightarrow (x \in A \cap B) \wedge (x \notin A \cap C) \\
&\Rightarrow x \in (A \cap B) - (A \cap C) \\
&\Rightarrow A \cap (B - C) \subseteq (A \cap B) - (A \cap C).
\end{align*}

Then, taking an element $y$ on the right-hand side, we have
\begin{align*}
&(y \in A \cap B) \wedge (y \notin A \cap C) \tag{definition of $-$} \\
&\Rightarrow (y \in A) \wedge (y \in B) \wedge (y \notin C) \tag{definition of $\cap$} \\
&\Rightarrow y \in (B - C) \\
&\Rightarrow y \in A \cap (B - C) \\
&\Rightarrow (A \cap B) - (A \cap C) \subseteq A \cap (B - C).
\end{align*}

Thus, we have shown that
\begin{align*}
A \cap (B - C) &\subseteq (A \cap B) - (A \cap C) \\
(A \cap B) - (A \cap C) &\subseteq A \cap (B - C)
\end{align*}
which means that $A \cap (B - C) = (A \cap B) - (A \cap C)$.
\end{proof}
\end{answer}

%
%
% Question 3 (done)
%
%
\newpage
\begin{exer}
Proposition 5.20(ii). If $A, B, C$ are any sets, then $A \times (B \cap C) = (A \times B) \cap (A \times C)$.
\end{exer}

\begin{proof}
Take an element $x$ on the left-hand side. We'll also use $i$, and $j$ to represent the ordered pair $(i, j)$ that $x$ constitutes. We can see that
\begin{align*}
i &\in A \tag{definition of $\times$} \\
j &\in B \cap C \Rightarrow (j \in B) \wedge (j \in C). \tag{definition of $\cap$}
\end{align*}

By the definition of $\times$, we know that
\begin{align*}
(i, j) \in A \times B \Rightarrow x \in A \times B \\
(i, j) \in A \times C \Rightarrow x \in A \times C.
\end{align*} This means we can say that $x \in (A \times B) \cap (A \times C)$ and therefore, \[ A \times (B \cap C) \subseteq (A \times B) \cap (A \times C). \]

Now we will examine an element $y$ from the right-hand side. We will use $k$, and $l$ to represent the ordered pair $(k, l)$ that $y$ constitutes. From the definition of $\cap$, we know that $(y \in A \times B) \wedge (y \in A \times C)$. We also know from the definition of $\times$ that
\begin{align*}
&(k, l) \in A \times B \Rightarrow (k \in A) \wedge (l \in B) \\
&(k, l) \in A \times B \Rightarrow l \in C.
\end{align*}

This means that $l \in B \cap C$. This means we can say \[ (k, l) \in A \times (B \cap C), \] i.e., $y \in A \times (B \cap C)$. From this we can conclude that \[ (A \times B) \cap (A \times C) \subseteq A \times (B \cap C). \]

To summarize, we have shown
\begin{align*}
A \times (B \cap C) &\subseteq (A \times B) \cap (A \times C) \\
(A \times B) \cap (A \times C) &\subseteq A \times (B \cap C)
\end{align*} and therefore \[ A \times (B \cap C) = (A \times B) \cap (A \times C). \]
\end{proof}

%
%
% Question 4 (done)
%
%
\newpage
\begin{exer}
Project 5.21(ii). $(A \times B) \cap (C \times D) = (A \cap C) \times (B \cap D).$ % (Prove or disprove. If you believe the equation is true, give a proof. If you believe it is not true for all sets, give a counterexample. You can try making up some random sets and working out each side of the equation to see if they match or not.)
\end{exer}

\begin{answer}
I claim this statement is true.

\begin{proof}
Take an element $x$ on the left-hand side. We will use $i$ and $j$ to represent the ordered pair $(i, j)$ that $x$ denotes. We can see that
\begin{align*}
&x \in A \times B \Rightarrow (i, j) \in A \times B \Rightarrow (i \in A) \wedge (j \in B) \\
&x \in C \times D \Rightarrow (i, j) \in C \times D \Rightarrow (i \in C) \wedge (j \in D).
\end{align*} This means that $i \in A \cap C$ and $j \in B \cap D$. This implies that the ordered pair $(i, j)$ is an element of $(A \cap C) \times (B \cap D).$ So, $x \in (A \cap C) \times (B \cap D)$. We can then conclude \[ (A \times B) \cap (C \times D) \subseteq (A \cap C) \times (B \cap D). \]

Now we will examine an element $y$ on the right-hand side. We will use $k$ and $l$ to the ordered pair $(k, l)$ that $y$ denotes. We can see that
\begin{align*}
&k \in A \cap C \Rightarrow (k \in A) \wedge (k \in C) \\
&l \in B \cap D \Rightarrow (l \in B) \wedge (l \in D).
\end{align*} This means that $(k, l) \in A \times B$ and $(k, l) \in C \times D$ and thus, $(k, l) \in (A \times B) \cap (C \times D)$. That is, \[ y \in (A \times B) \cap (C \times D). \] We can then conclude \[ (A \cap C) \times (B \cap D) \subseteq (A \times B) \cap (C \times D). \]

We have shown that
\begin{align*}
(A \times B) \cap (C \times D) &\subseteq (A \cap C) \times (B \cap D) \\
(A \cap C) \times (B \cap D) &\subseteq (A \times B) \cap (C \times D),
\end{align*} and therefore \[ (A \times B) \cap (C \times D) = (A \cap C) \times (B \cap D). \]
\end{proof}
\end{answer}

%
%
% Question 5
%
%
\newpage
\begin{exer}
Find $\sum_{j = 0}^k f^2_j$ = (your answer), where the are Fibonacci numbers as defined in the textbook. Prove your answer.

Your answer will have a clear statement: (your answer). Then, a proof of your answer.

For your proof, use induction.

For example, since the Fibonacci numbers start with , you will get values starting with 0, 1, 1, 2, 3, 5, 8, ..., you will get values starting with $0^2 = 0, 0^2 + 1^2 = 1, 1^2 + 1^2 = 2, 1^2 + 1^2 + 2^2 = 6$, and so on.

Hint: Work out some of the values and look for a pattern. It might be helpful to look at factorizations of the values.
\end{exer}

\begin{answer}
After inspecting the series/summation, I propose that

$$
\sum_{j = 0}^{k} f^2_j = f_k \cdot f_{k + 1}
$$

for $k \geq 1$.

\begin{proof}
Let $P(k)$ be defined as $\sum_{j = 0}^{k} f^2_j = f_k \cdot f_{k + 1}$ for $k \geq 1$. As a base case, we will examine $P(1)$:

\begin{align*}
\sum_{j = 0}^1 f^2_j = 0^2 + 1^2 = 0 + 1 &= 1 \\
f_1 \cdot f_{2} = 1 \cdot 1 &= 1.
\end{align*}

We know $P(k)$ is true for some $k$. Let us assume $P(k)$ is true, and use this to show that $P(k + 1)$ is also true. That is, we intend to show that $P(k + 1)$,
$$
\sum_{j = 0}^{k + 1} f^2_j = f_{k + 1} \cdot f_{k + 2}
$$ is true.
\end{proof}

\end{answer}

\end{document}
