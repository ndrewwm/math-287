\documentclass[12pt,oneside]{amsart}

\title{Math 287 Homework 5}
\author{Andrew Moore}
\date{October 10, 2021} % the due date of the homework

\usepackage[T1]{fontenc}
\usepackage{amsmath,amsfonts,amssymb,amsthm}
\usepackage[letterpaper,margin=1.5in]{geometry}
\usepackage{fancyhdr}
\pagestyle{fancy}
\usepackage{enumitem}

% Extra space between lines
\linespread{2.4}

\theoremstyle{remark}
\newtheorem{exer}{Exercise}
\newtheorem{claim}{Claim}[exer]

\newcommand{\bfN}{\mathbf{N}}
\newcommand{\bfZ}{\mathbf{Z}}


\newenvironment{answer}{\bigskip\noindent\emph{Answer.}}{\hfill$\diamond$\newline}

\begin{document}
\maketitle

%
%
% Question 1
%
%
\begin{exer}
Show some of your favorite equations in \emph{inline} and \textbf{display} mathematics.
\end{exer}

\begin{answer}
Some people like small equations such as $e^{\pi i}+1 = 0$
or \( a^2+b^2 = c^2 \).

My favorite small equation is $...$.

Other people like big equations like $$ x = \frac{-b \pm \sqrt{b^2-4ac}}{2a} $$
and \[ \int x^n \, dx = \frac{1}{n+1} x^{n+1} + C . \]

My favorite big equation is
\[ ... \]
\end{answer}

%
%
% Question 2
%
%
\newpage
\begin{exer}
Project 5.16(ii). Prove or give a counterexample: For all sets $A,B,C$,
$A \cap (B-C) = (A \cap B) - (A \cap C)$.
\end{exer}

% The symbol for intersection of sets is typed \cap.
% The symbol for union of sets is typed \cup.

\begin{answer}
-enter your answer here-
\end{answer}

%
%
% Question 3
%
%
\begin{exer}
Proposition 5.20(ii).
\end{exer}

%
%
% Question 4
%
%
\begin{exer}
Project 5.21(ii).

(Prove or disprove. If you believe the equation is true, give a proof. If you believe it is not true for all sets, give a counterexample. You can try making up some random sets and working out each side of the equation to see if they match or not.)
\end{exer}

%
%
% Question 5
%
%
\begin{exer}
Find $\sum_{j = 0}^k f^2_j$ = (your answer), where the are Fibonacci numbers as defined in the textbook. Prove your answer.

Your answer will have a clear statement: (your answer). Then, a proof of your answer.

For your proof, use induction.

For example, since the Fibonacci numbers start with , you will get values starting with 0, 1, 1, 2, 3, 5, 8, ..., you will get values starting with $0^2 = 0, 0^2 + 1^2 = 1, 1^2 + 1^2 = 2, 1^2 + 1^2 + 2^2 = 6$, and so on.

Hint: Work out some of the values and look for a pattern. It might be helpful to look at factorizations of the values.
\end{exer}

\end{document}
