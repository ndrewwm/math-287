% Reading assignments & Resources from Dr. Teitler:
% Read Chapter 3 and Chapter 4 of the textbook.
% A Guide to Writing Mathematics https://web.cs.ucdavis.edu/~amenta/w10/writingman.pdf
% Getting Better at Proofs https://math.stackexchange.com/questions/7743/getting-better-at-proofs
% Unexpected Mathematical Images https://mathoverflow.net/questions/178139/examples-of-unexpected-mathematical-images
% AMS Graduate Programs http://www.ams.org/education/pre-grad

\documentclass[12pt,oneside]{amsart}

\title{Math 287 Homework 2}
\author{Andrew Moore}
\date{September 10, 2021} % the due date of the homework

\usepackage[T1]{fontenc}
\usepackage{amsmath,amsfonts,amssymb,amsthm}
\usepackage[letterpaper,margin=1.5in]{geometry}
\usepackage{fancyhdr}
\pagestyle{fancy}

% Extra space between lines
\linespread{2.4}

\theoremstyle{remark}
\newtheorem{exer}{Exercise}
\newtheorem{claim}{Claim}[exer]

\newcommand{\bfN}{\mathbf{N}}
\newcommand{\bfZ}{\mathbf{Z}}


\newenvironment{answer}{\bigskip\noindent\emph{Answer.}}{\hfill$\diamond$\newline}

\begin{document}
\maketitle

\newpage
\begin{exer}
Proposition 2.7(iv). Let $m,n,p,q \in \bfZ$.
If $m < n$ and $p < 0$ then $np < mp$.
\end{exer}

\begin{proof}
We are told that $n - m \in \bfN$, and $0 - p \in \bfN$ (i.e., $-p \in \bfN$, based on Axiom 2.1.4). We will show that $mp - np \in \bfN$. Rearranging $mp - np$ we have:
\begin{align*}
mp - np &= p(m - n) \\
        &= p(m(-1)(-1) + n(-1)) \\
        &= p(-1)(m(-1) + n) \\
        &= -p((-m) + n) \\
        &= -p(n - m) \in \bfN.
\end{align*}

The series of steps above demonstrate that $mp - np$ can be expressed as two terms we know to be natural numbers (based on our initial assumptions): $-p$ and $(n - m)$. Axiom 2.1.2 tells us that the product of any two natural numbers is also in $\bfN$. Therefore, $(mp - np)$ must be in $\bfN$, i.e. $np < mp$.
\end{proof}

\newpage
\begin{exer}
Proposition 2.12(iii). For all $m,n,p \in \bfZ$,
if $p < 0$ and $mp < np$ then $n < m$.
\end{exer}

\begin{proof}
We are told we can assume the following: $0 - p \in \bfN$, and $np - mp \in \bfN$. We must show that $n < m$, i.e. $m - n \in \bfN$. Rearranging $np - mp \in \bfN$, we have:
\begin{align*}
np - mp &= p(n - m) \\
        &= p(n(-1)(-1) + m(-1)) \\
        &= p(-1)(n(-1) + m) \\
        &= -p(-n + m) \\
        &= -p(m - n) \in \bfN.
\end{align*}

We can reexpress $0 - p \in \bfN$ as $-p \in \bfN$ (which we know from Axiom 2.1.4). From the series of equations above, we've seen that $np - mp \in \bfN$ can be expressed as $-p(m - n) \in \bfN$. Axiom 2.1.2 tells us that the product of any two numbers in $\bfN$ is also in $\bfN$. Therefore, $(m - n)$ must be in $\bfN$, i.e. $n < m$.

\end{proof}

\newpage
\begin{exer}
Proposition 2.26. In this problem, the textbook gives a proof. Your homework is to rewrite the proof in more detail.

Imagine a student in the class is confused by the proof. Rewrite the proof in a way that would make sense and be clear for a confused student.
\end{exer}

\newpage
\begin{exer}
Project 2.28. Determine for which natural numbers $k^2 - 3k \geq 4$
and prove your answer.
\end{exer}

\begin{answer}
\begin{claim}
$k^2 - 3k \geq 4$ for $k \geq 4$.
\end{claim}

\begin{proof}
We are given the inequality, $k^2 - 3k \geq 4$, and are asked to find (and justify) the values of $k$ that make the inequality true. We will prove the claim $k^2 - 3k \geq 4$ for $k \geq 4$ using induction. First, note that $k \leq 3$ does not satisfy the inequality.

For $k = 3$:
\begin{align*}
    k^2 - 3k \geq 4 \Rightarrow
(3)^2 - 3(3) \geq 4 \Rightarrow
       9 - 9 \geq 4 \Rightarrow
           0 \geq 4
\end{align*}

The statement $0 \geq 4$ is \emph{false}, because $0 \neq 4$ and $0 - 4 \notin \bfN$. It is trivial to demonstrate the same result for $k = 1$ and $k = 2$, and this is left to the reader. Assuming that $k$ can't be $\leq 3$, let us redefine $k$ to be $k = 3 + j$, such that $j \in \bfN$ and $j \geq 1$. We will now prove the base case ($j = 1$).

For $k = 3 + 1 = 4$:
\begin{align*}
    k^2 - 3k \geq 4 \Rightarrow
(4)^2 - 3(4) \geq 4 \Rightarrow
     16 - 12 \geq 4 \Rightarrow
           4 \geq 4
\end{align*}

This last statement is true ($4 = 4$). We will now show that the results apply for all j, by proving this is true for $k = 4 + (j + 1)$.

For $k = 4 + (j + 1) = 5 + j$:
\begin{align*}
   (k + 1)^2 - 3(k + 1) &\geq 4 \\
(k^2 + 2k + 1) - 3k - 3 &\geq 4 \\
            k^2 - k - 2 &\geq 4 \\
(5 + j)^2 - (5 + j) - 2 &\geq 4 \\
 25 + 10j + j^2 - j - 7 &\geq 4 \\
          j^2 + 9j + 18 &\geq 4
\end{align*}

The final statement $j^2 + 9j + 18 - 4 \in \bfN$ is true because there is no value of $j$ in $\bfN$ that could cause the value of the statement to be $\leq 0$.
\end{proof}
\end{answer}

\end{document}
